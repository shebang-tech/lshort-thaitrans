% This file starts from page 1 of the original book.

\chapter[สิ่งที่คุณควรรู้\ldots]{สิ่งที่คุณควรรู้ก่อนที่เราจะเริ่มเรียนการใช้งาน \LaTeXe}
ส่วนแรกของบทนี้ เราจะเริ่มด้วยการเล่าถึงประวัติและปรัชญาของ \LaTeXe{}
แล้วในส่วนหลังของบท จึงจะตามด้วยเรื่องเกี่ยวกับโครงสร้างพื้นฐานของ \LaTeX{} หลังจาก%
อ่านบทแรกนี้จบแล้ว คุณจะพอทราบคร่าวๆว่า \LaTeX{} ทำงานอย่างไร ซึ่งจำเป็นในการทำ%
ความเข้าใจในบทอื่นๆต่อไป

\section{ชื่อนี้ท่านได้แต่ใดมา}

\subsection{\TeX}

\TeX เป็นซอฟต์แวร์ที่ถูกสร้างขึ้นมาโดยคุณ Donald E. Knuth เพื่อใช้ในการเรียงพิมพ์ตัวหนังสือ
และสูตรคณิตศาสตร์ คุณ Knuth เริ่มพัฒนา \TeX{} ในปี ค.ศ.\ 1977 เพื่อทดลองดูว่า%
งานพิมพ์ด้วยระบบดิจิตัลจะมีคุณภาพได้สูงถึงขั้นไหน โดยเขาหวังว่าเขาจะสามารถแก้ไขปัญหาใน%
ยุคนั้นที่งานพิมพ์ด้วยระบบดิจิตัลมักมีคุณภาพต่ำ ซึ่งส่งผลกระทบต่องานพิมพ์หนังสือและบทความ%
ที่เขากำลังเขียนอยู่ โปรแกรม \TeX{} ที่พวกเรากำลังใช้อยู่ทุกวันนี้ถูกพัฒนาเสร็จและเปิดให้คน%
ทั่วไปได้ใช้งานเป็นครั้งแรกในปี ค.ศ.\ 1982 และ ต่อมาในปี ค.ศ.\ 1989 คุณ Knuth ได้%
ออกเวอร์ชั่นใหม่ที่มีการแก้ไขปรับปรุงเพียงเล็กน้อยเพื่อให้ \TeX{} รองรับตัวอักขระแบบ 8-bit
และภาษาอื่นๆนอกเหนือจากภาษาอังกฤษได้ โปรแกรม \TeX{} นั้นมีชื่อเสียงโด่งดังมากในเรื่อง%
ความมีเสถียรภาพสูง, สามารถรันได้บนคอมพิวเตอร์มากมายหลายรุ่น หลายยี่ห้อ, และว่ากันว่า%
ไม่มีบั๊ก เลขเวอร์ชั่นของ \TeX{} จะเป็นเลขทศนิยมที่วิ่งเพิ่มขึ้นเข้าใกล้ค่า $\pi$ ไปเรื่อยๆ
และขณะนี้มันมีเวอร์ชั่น $3.141592653$

\shbtoaddmoretranslation
