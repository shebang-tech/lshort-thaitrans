% This file starts from page 1 of the original book.

\chapter[สิ่งที่คุณควรรู้\ldots]{สิ่งที่คุณควรรู้ก่อนที่เราจะเริ่มเรียนการใช้งาน \LaTeXe}
ส่วนแรกของบทนี้ เราจะเริ่มด้วยการเล่าถึงประวัติและปรัชญาของ \LaTeXe{}
แล้วในส่วนหลังของบท จึงจะตามด้วยเรื่องเกี่ยวกับโครงสร้างพื้นฐานของ \LaTeX{} หลังจาก%
อ่านบทแรกนี้จบแล้ว คุณจะพอทราบคร่าวๆว่า \LaTeX{} ทำงานอย่างไร ซึ่งจำเป็นในการทำ%
ความเข้าใจในบทอื่นๆต่อไป

\section{ชื่อนี้ท่านได้แต่ใดมา}

\subsection{\TeX}

\TeX เป็นซอฟต์แวร์ที่ถูกสร้างขึ้นมาโดยคุณ Donald E. Knuth เพื่อใช้ในการเรียงพิมพ์ตัวหนังสือ
และสูตรคณิตศาสตร์ คุณ Knuth เริ่มพัฒนา \TeX{} ในปี ค.ศ.\ 1977 เพื่อทดลองดูว่า%
งานพิมพ์ด้วยระบบดิจิตัลจะมีคุณภาพได้สูงถึงขั้นไหน โดยเขาหวังว่าเขาจะสามารถแก้ไขปัญหาใน%
ยุคนั้นที่งานพิมพ์ด้วยระบบดิจิตัลมักมีคุณภาพต่ำ ซึ่งส่งผลกระทบต่องานพิมพ์หนังสือและบทความ%
ที่เขากำลังเขียนอยู่ โปรแกรม \TeX{} ที่พวกเรากำลังใช้อยู่ทุกวันนี้ถูกพัฒนาเสร็จและเปิดให้คน%
ทั่วไปได้ใช้งานเป็นครั้งแรกในปี ค.ศ.\ 1982 และ ต่อมาในปี ค.ศ.\ 1989 คุณ Knuth ได้%
ออกเวอร์ชั่นใหม่ที่มีการแก้ไขปรับปรุงเพียงเล็กน้อยเพื่อให้ \TeX{} รองรับตัวอักขระแบบ 8-bit
และภาษาอื่นๆนอกเหนือจากภาษาอังกฤษได้ โปรแกรม \TeX{} นั้นมีชื่อเสียงโด่งดังมากในเรื่อง%
ความมีเสถียรภาพสูง, สามารถรันได้บนคอมพิวเตอร์มากมายหลายรุ่น หลายยี่ห้อ, และว่ากันว่า%
ไม่มีบั๊ก เลขเวอร์ชั่นของ \TeX{} จะเป็นเลขทศนิยมที่วิ่งเพิ่มขึ้นเข้าใกล้ค่า $\pi$ ไปเรื่อยๆ
และขณะนี้มันมีเวอร์ชั่น $3.141592653$

\TeX{} อ่านออกเสียงว่าเหมือนคำว่า ``Tech,'' โดยตัว ``ch'' จะมีเสียงเหมือนคำว่า
``Ach'' ในภาษาเยอรมัน\footnote{ในภาษาเยอรมันจะมีการออกเสียง ``ch`` อยู่ 2 แบบ
แบบแรกคือแบบที่เหมือน ``ch'' ในคำว่า ``Ach'' ซึ่งมีเสียงหนักกว่า ส่วนอีกแบบคือแบบที่%
เหมือน ``ch'' ในคำว่า ``Pech'' ซึ่งมีเสียงอ่อนกว่า บางท่านอาจจะคิดว่าจะเหมาะกว่าที่จะ%
ออกเสียง ``ch'' ด้วยเสียงอ่อน ทั้งนี้คุณ Knuth ได้กล่าวถึงเรื่องการออกเสียงคำว่า \TeX{}
ไว้บน Wikipedia ภาษาเยอรมันเอาไว้ว่า \emph{ผมไม่โกรธคนที่ออกเสียงคำว่า \TeX{}
ในแบบที่เขาชอบหรอก\ldots{} และในประเทศเยอรมันก็มีคนหลายคนที่ใช้เสียง ``ch''
แบบอ่อน เพราะตัว X ตามหลังสระ e ซึ่งแตกต่างจากตัว X ที่ตามหลังสระ a
ซึ่งจะออกเสียงแบบหนัก ส่วนในรัสเซียนั้น `tex` เป็นคำที่ใช้กันโดยทั่วไป และออกเสียงว่า
`tyekh' แต่ผมคิดว่าการออกเสียงที่ดีที่สุดสำหรับคำว่า \TeX{} นี้น่าจะเป็นการออกเสียงคำนี้%
ด้วยภาษากรีก ที่จะออกเสียง ch เหมือนในคำว่า ach และ Loch ซึ่งจะเป็นเสียงที่ดังกว่า}}
หรือ ``Loch'' ในภาษาสกอต และเจ้าตัว ``ch'' นี้มีต้นกำเนิดมาจากตัวอักษรในภาษากรีก คือ
ตัว X ที่ในอักษรโรมันเราเขียนว่า ``ch'' หรือ ``chi'' นอกจากนี้ \TeX{} ยังเป็นพยางค์%
แรกของคำว่า teqnik'h\footnote{ผู้แปล: ผู้แปลยังไม่รู้วิธีพิมพ์อักขระกรีกด้วย \LaTeX และ
ยังไม่สะดวกค้นหาวิธี จึงต้องพิมพ์ด้วย ASCII แบบนี้ไปก่อน ต้องขออภัยมา ณ ที่นี้} ในภาษากรีก%
อีกด้วย สำหรับในระบบ ASCII นั้น \TeX{} จะเขียนแบบนี้ TeX

\subsection{\LaTeX}

\LaTeX{} ช่วยให้นักเขียนสามารถพิมพ์ผลงานของเขาออกมาด้วยระบบการพิมพ์ที่มีคุณภาพสูงที่สุด
\LaTeX{} ถูกพัฒนาขึ้นเป็นครั้งแรกโดย นาย \index{Lamport, Leslie}Leslie
Lamport~\cite{manual}  มันใช้ \TeX{} เป็นระบบเรียงพิมพ์ (ที่คนมักกล่าวกันว่า
\LaTeX{} คือ มาโครของ \TeX{} แต่ทุกวันนี้ผู้ที่คอยดูแลงานพัฒนาซอฟต์แวร์ \LaTeX{}
คนปัจจุบันคือ \index{Mittelbach, Frank}นาย Frank Mittelbach

\LaTeX{} ออกเสียงว่า ``เล-เท็ค'' หรือ ``ลา-เท็ค'' หากคุณจะพิมพ์ \LaTeX{}
ใส่โปรแกรมใดๆที่รองรับอินพุทแบบ ASCII ให้พิมพ์ว่า LaTeX ส่วน \LaTeXe{}
ออกเสียงว่า ``เล-เท็ค ทู อี'' และพิมพ์ว่า LaTeX2e

\section{ความรู้พื้นฐาน}

\subsection{นักเขียนหนังสือ, นักออกแบบหนังสือ และนักเรียงพิมพ์}

การตีพิมพ์หนังสือใดๆ นักเขียนจะส่งงานเขียนของเขาให้กับสำนักพิมพ์%
แล้ว Book Designer ที่นั่นจะออกแบบการจัดวางเนื้อหาต่างๆในหนังสือ (เช่น
ความกว้างของแต่ละคอลัมน์, ฟอนท์ที่จะใช้, ระยะห่างระหว่างบรรทัดทั้งก่อนและหลังหัวข้อต่างๆ%
,~\dots) โดยงานออกแบบนี้จะเรียกว่า Manuscript ซึ่งจะถูกส่งต่อไปยัง Typesetter
ที่มีหน้าที่ในการเรียงพิมพ์หนังสือตามที่ถูกกำหนดมาใน Manuscript

\shbtoaddmoretranslation
